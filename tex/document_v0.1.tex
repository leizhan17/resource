\documentclass{ctexart}
\usepackage[colorlinks, linkcolor=red, anchorcolor=blue, citecolor=green]{hyperref}

\begin{document}
\title{学习资源推荐v0.1}
\author{EESAST\ (王敏虎、武楚涵、李润桐、杨怿飞、王启睿、陈誉博、黄秀峰)}
\date{\today}
\maketitle

\section{Git}

\subsection{\href{https://www.liaoxuefeng.com/wiki/0013739516305929606dd18361248578c67b8067c8c017b000}{廖雪峰Git教程}}
一个非常简单易懂的Git入门教程

\subsection{\href{http://iissnan.com/progit/html/zh/ch1_0.html}{Git权威指南中文手册}}
一个挺详细的Git中文手册

\section{Linux}

\subsection{鸟哥的Linux私房菜}
\href{http://linux.vbird.org}{网页版},非常经典的一本书,很好的Linux学习资料

\subsection{菜鸟教程}
\href{http://www.runoob.com/linux/linux-tutorial.html}{菜鸟教程-Linux教程},内容不多,简单易懂,花一两天快速看完就能了解Linux的基本日常使用了


\section{Python}
\subsection{\href{https://www.liaoxuefeng.com/wiki/0014316089557264a6b348958f449949df42a6d3a2e542c000}{廖雪峰Python教程}}
一个通俗易懂但内容十分丰富的Python教程,涵盖了常见的编程方面的内容,完整看完还能意识到我们的程序设计课学的内容有多薄弱

\subsection{Python Cookbook}
\begin{itemize}
	\item 豆瓣评分9.2分(中文版8.5分),介绍了一些编程中的奇技淫巧,对一些常用库也有所涉及。
	\item 豆瓣链接 https://book.douban.com/subject/20491078/
\end{itemize}

\subsection{Beginning Python: From Novice to Professional}
\begin{itemize}
	\item 由浅入深,通俗易懂
 	\item 有中文版,网上PDF一大把,学校图书馆也有
\end{itemize}   
 
\subsection{Python for Data Analysis}

\begin{itemize}
	\item 豆瓣评分8.3分(中文版8.5分),介绍了一些与科学计算、数据分析相关的库,如numpy、matplotlib 等。
	\item 豆瓣链接 https://book.douban.com/subject/25779298/
\end{itemize}

\section{Web Development}
\subsection{HTML}
\begin{itemize}
\item \href{http://www.w3school.com.cn/}{W3School}
\item \href{https://developer.mozilla.org/en-US/}{MDN}
\end{itemize}
\subsection{JavaScript}

\begin{itemize}
\item 犀牛书
\item JavaScript 函数式编程
\end{itemize}

\subsection{CSS}
\begin{itemize}
\item \href{https://css-tricks.com/}{CSS Tricks}
\end{itemize}

\subsection{Python > Django}
\begin{itemize}
\item \href{https://docs.djangoproject.com/}{官方文档} 遇到问题多查查文档总是好的
\item \href{https://code.ziqiangxuetang.com/django/django-tutorial.html}{自强学堂Django教程} 内容挺丰富的,能涵盖Django大部分常见功能
\end{itemize}

\subsection{Node.js}

\begin{itemize}
\item \href{https://nodejs.org/en/}{官方文档} Node.js的中文社区非常强健,大量node.js的布道者基本已经完成了node.js文档的翻译和整理。
\item \href{https://book.douban.com/subject/26937390/}{Node.js硬实战} 适合在看完文档并自行完成一至二个demo后阅读提升。
\item \href{https://www.coursera.org/specializations/full-stack-mobile-app-development}{web development} coursera的Full Stack Web Development 专项课程。提供了比较全面的入门指导。
\end{itemize}

\section{APP Development}

\subsection{Android}

\begin{itemize}
	\item \href{http://hukai.me/android-training-course-in-chinese/index.html}{Android Training官方课程} :一个手把手教你入门Android开发的教程
    \item {《Thinking In Java》} :Java是开发Android使用的编程语言(没用过Kotlin,不评价),所以好好学学Java还是有必要的(当然你也可以选择好好学习JNI)。电子书网上自己应该能搜到,我就不放链接了。
    
\end{itemize}
\section{Machine Learning/Deep Learning/Reinforcement learning}


\subsection{CS229}

\begin{itemize}
	\item Andrew Ng的经典Machine Learning课程,更多的基于概率统计有关的数学推导,反正是一个经典公开课,好不好就看你自己口味了
	\item 顺便在这推一下《统计学习方法》,中文书,两百多页吧大概,虽然内容不多不深,但也比较全面,适合机器学习入门
\end{itemize}


\subsection{CS231n}

\begin{itemize}
	\item 斯坦福大学的一个非常著名的深度学习与计算机视觉课程,由Li Feifei及其两位高徒授课
	\item \href{https://zhuanlan.zhihu.com/p/21930884?refer=intelligentunit}{cs231n官方笔记中文翻译},原视频可以在B站看到
\end{itemize}

\subsection{CS224n}
\begin{itemize}
	\item 斯坦福大学公开课,Deep Learning for NLP,从基本的语言模型、词向量等概念出发,介绍了深度学习在自然语言处理中的应用,并介绍了各种不同任务的常见模型。
    \item \href{http://web.stanford.edu/class/cs224n/}{官方网站}
\end{itemize}
\subsection{CS294}

\begin{itemize}
	\item 伯克利大学的增强学习课程,公开课视频在网上能搜到,下面给的“知乎资料”里也有提供视频入口
    \item \href{https://zhuanlan.zhihu.com/p/24721292}{来自知乎的一点学习资料} 
\end{itemize}

\subsection{David Silver的强化学习课程}

\begin{itemize}
	\item 这是一个非常入门级的增强学习课程,\href{http://techtalks.tv/talks/deep-reinforcement-learning/62360}{视频链接} 
    \item \href{https://zhuanlan.zhihu.com/reinforce}{来自知乎的一点学习资料} 
\end{itemize}

\subsection{《Neural Networks and Deep Learning》}
\begin{itemize}
    \item Michael Nielsen 写的一部轻量级的 Deep Learning 入门资料,虽然没有下面要介绍的《Deep Learning》有信仰,但是非常适合用作从零开始的入门读物。该教程以MNIST为例,介绍了深度学习中几乎所有重要的基本概念。全书只有200页左右,读起来非常轻松。
    \item \href{https://github.com/zhanggyb/nndl/releases}{中文版PDF}
    
\end{itemize}
\subsection{《Deep Learning》}

\begin{itemize}
	\item Benjio, Goodfellow 和 Courville 三位大佬联袂巨作,从最基础的矩阵运算和概率相关知识开始慢慢深入。涉及内容较广,内容通俗易懂,适合于整体掌握深度学习整个发展框架,配合与原文阐述内容相关的paper阅读效果更佳。
    \item 英文版中文版网上都一搜一堆,\href{https://github.com/exacity/deeplearningbook-chinese}{中文版链接}
\end{itemize}


\subsection{PRML等有关书籍}

\begin{itemize}
	\item 包括PRML(Pattern Recognition and Machine Learning)、MLAPP(Machine Learning: A Probabilistic Prospective)等大量有关于机器学习的背后知识的理论基础与数学推导,更有助于对于ML后的理论进行更深层次的理解
    \item \href{https://github.com/csuldw/MachineLearning/blob/master/doc/PRML\_Zh.pdf}{PRML传送门}
    \item \href{http://vdisk.weibo.com/s/jTT1B}{MLAPP传送门}
\end{itemize}

\subsection{Udacity}

\begin{itemize}
\item 价格非常昂贵(万级)
\end{itemize}
\end{document}